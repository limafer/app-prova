
    \documentclass[12pt,a4paper]{article}
    \usepackage[utf8]{inputenc}
    \usepackage{amsmath}
    \usepackage{amsfonts}
    \usepackage{amssymb}
    \usepackage{graphicx}
    %\usepackage{tikz}
    %\usepackage{tikz-3dplot}
    \usepackage{enumerate}
    %\usetikzlibrary{intersections}
    \usepackage[left=2.0cm,top=1.5cm,right=2.0cm]{geometry}
    \newcommand{\edo}[3]{$#1\,{\rm{d}}x#2\,{\rm{d}}y=#3$}
    \thispagestyle{empty}
    
    \everymath{\displaystyle}
    

\begin{document}
\begin{center}
\large LISTA DE EXERCÍCIOS DE ALGEBRA LINEAR\\[1mm]
{\large\bf Departamento de Matemática}\\[1mm] \textit{Prof. Lindeval: lindeval.ufrr@gmail.com}\\\end{center}

\noindent\rule{17.0cm}{0.7mm}\\[0.5cm]
\noindent
1) Dados os vetores $e_1=(1,0)$  e $e_2=(0,1),$ mostre que o conjunto $\{e_1,e_2\}$ é uma base para $\mathbb{R}^2.$\\[2mm]
\noindent
2) Defina conjunto LI e conjunto LD.\\[2mm]
\vfill\hfill\bf{\textit{Boas Atividades!}}
\end{document}